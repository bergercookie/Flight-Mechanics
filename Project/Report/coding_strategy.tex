\section{Programming Strategy}
\label{sec:coding}

Even though comments exist at every phase of the coding procedure, it is useful to give
an insight on the programming strategy and the steps followed to reach the final outcome.

\subsection{Excess Thrust, SEP Diagrams}

The steps followed to produce the necessary graphs are the following
\begin{description*}
    \item [Global Variables Initialization] \hfill \\
        Set useful variables as constants so that they are accessible in all the modules
    \item [Necessary Structures Initialization] \hfill \\
        In order to have a more concrete way of handling the data, struct matlab types were 
        used to store the values of Excess Thrust, Sep and the alfa values, for each altitude.
        All the data derived from looping for each altitude and mach number are eventually stored 
        in the $curves$ struct which contains 4 fields, \textit{alt, thrusts, powers, and alfas}.
        Thrusts and powers are 2D arrays which correspond to each altitude while the 
        alfas are 1D arrays corresponding to each altitude.
        For simpler data such as the range of altitudes and mach numbers as well 
        as the plotting parameters (style, color) for each curve matlab arrays were used
    \item [Plotting the Data] \hfill \\
        The data stored in the curves struct are then plotted first in a 2D plot (Mach - Tex)
        and then in a contour plot (Mach, altitudes, Sep). The range given as a special 
        parameter in the contour plot is also important here as it defines the parts of 
        the graph which are visible as well as the overall detail of it.
    \item [Plotting the envelope limits] \hfill \\
        In order to plot the envelope limits as curves in the SEP Graph for both 
        the alfa limit and the dynamic pressure,
        we must first gather the points closer to the contours
        and then filter out the zeros in the end of the arrays. In the end we fit a 
        least-squares polynomial function through the points.
\end{description*}

\noindent To simplify the coding procedure, essential parts of the program were put into seperate modules / functions:
\begin{description*}
    \item[TexSep] \hfill \\
        Function for calculating the Excess thrust the SEP as well as the angle of attack. The function
        is provided with the altitude, the mach, the mass and the gamma angle as inputs as well as some global variables
        (see function documentation for more details).
    \item[find\_a] \hfill \\
        Provided the thrust the mass and the gamma angle as input arguements, find\_a
        solves the nonlinear equation $Tsin(a + \epsilon{\tau} + L(a) - mgocos(gamma)$ 
        in order to calculate a value for the alfa angle.
\end{description*}

\subsection{Optimization Part}
