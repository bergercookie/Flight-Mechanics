\section{Introduction}

As the jet era started, Sweden foresaw the need for a jet fighter that could intercept bombers at high 
altitude and also successfully engage fighters. Although other interceptors such as the 
US Air Force's F-104 Starfighter were being conceived during the same period, 
Saab's "Draken" would have to undertake a combat role unique to Sweden. 
Other demanding requirements were the capability to operate from reinforced public roads 
used as part of wartime airbases, and for refuelling/rearming to be 
carried out in no more than ten minutes, by conscripts with minimal training. 
\textbf{In September 1949, the Swedish Defence Material Administration issued a request for a fighter/interceptor aircraft, and work began at Saab the same year}

\textit{Regarding the aerodynamic design of the J35 Draken the two major options were swept wings and delta wings \cite{dorr_aerofax_1987}.}
The question was quickly resolved by the initial studies which had called for the
exploration of a swept wing configuration. In short order it  was determined tha
in consideration of all other parameteres placed upon the design, 
the swept wing's aerodynamic drag at high Mach numbers was too high, 
and its configuration requirements dictated that the fuselage have insufficient 
volume for equipment, fuel and armament.
The Delta wing on the other hand showed great promise following initial tunnel 
tests. The pure delta soon was ruled out, however, as it suffered from center 
of gravity and center of pressure anomalies that were difficult to alleviate. 
A derivative, however, often referred to as \textit{the double delta}, proved much more flexible. 
In general the double delta was found to offer the attributes of:
\begin{itemize*}
    \item reduced frontal area while permitting optimal wing area
    \item More favorable wing sweep angles on the center wing section
    \item Center of gravity and center of pressure being closer to each other
    \item More favorable area distribution
    \item Low supersonic drag
    \item Favorable low speed drag
    \item Strong and stiff fail safe structure
    \item Being able to place the air intakese farther forward
\end{itemize*}


