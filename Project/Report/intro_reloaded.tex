\section{Introduction}

The \textit{SAAB J 35 Draken} was originally conceived as a replacement for the Swedish
Air Force’s venerable \textit{J 29 Tunnan}, an aircraft that was the equivalent of the
F-86 Sabre and MiG-15 in its capabilities, range, performance and load.

Regarding the aerodynamic design of the J35 Draken the two major options
to chose from were swept wings and delta wings \cite{dorr_aerofax_1987}.
The question was quickly resolved by the initial studies which had called
for the exploration of a swept wing configuration. In short order it  was
determined tha in consideration of all other parameteres placed upon the
design, the swept wing's aerodynamic drag at high Mach numbers was too high,
and its configuration requirements dictated that the fuselage have
insufficient volume for equipment, fuel and armament. The pure delta was
also ruled out, however, as it suffered from center of gravity and center of
pressure anomalies that were difficult to alleviate.  A derivative, however,
often referred to as \textit{the double delta}, proved much more flexible.

In general the double delta was found to offer the attributes of:
\begin{itemize*}
    \item Reduced frontal area while permitting optimal wing area
    \item More favorable wing sweep angles on the center wing section
    \item Center of gravity and center of pressure being closer to each other
    \item More favorable area distribution
    \item Low supersonic drag
    \item Favorable low speed drag
    \item Strong and stiff fail safe structure
    \item Being able to place the air intakese farther forward
\end{itemize*}

The plane’s fuselage was a  predictable tube with the engine mounted inside and
the cockpit at the front and a vertical stabilizer attached to the
tail~\cite{riseofDraken}.  The
pointed nose cone contained a radar system and the air intakes for the engine
were on either side of the cockpit at the forward point of the wing root.  The
double-delta began at the air intakes — for roughly two thirds of way toward the
tips, the sweep back measured an incredibly sharp 80 degrees.  This allowed the
plane to achieve design speeds that were in excess of Mach 2.0.  However, with
such a sharp sweep, it was recognized that the plane would be seriously lacking
in maneuverability.  Thus, the last one third of the wings toward the tips
carried a completely different sweep angle, much shallower at 60 degrees.  This
brought excellent maneuverability and enhanced control at low speeds.
